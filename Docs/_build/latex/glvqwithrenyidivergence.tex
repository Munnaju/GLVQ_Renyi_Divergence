%% Generated by Sphinx.
\def\sphinxdocclass{report}
\documentclass[letterpaper,10pt,english]{sphinxmanual}
\ifdefined\pdfpxdimen
   \let\sphinxpxdimen\pdfpxdimen\else\newdimen\sphinxpxdimen
\fi \sphinxpxdimen=.75bp\relax
\ifdefined\pdfimageresolution
    \pdfimageresolution= \numexpr \dimexpr1in\relax/\sphinxpxdimen\relax
\fi
%% let collapsable pdf bookmarks panel have high depth per default
\PassOptionsToPackage{bookmarksdepth=5}{hyperref}

\PassOptionsToPackage{warn}{textcomp}
\usepackage[utf8]{inputenc}
\ifdefined\DeclareUnicodeCharacter
% support both utf8 and utf8x syntaxes
  \ifdefined\DeclareUnicodeCharacterAsOptional
    \def\sphinxDUC#1{\DeclareUnicodeCharacter{"#1}}
  \else
    \let\sphinxDUC\DeclareUnicodeCharacter
  \fi
  \sphinxDUC{00A0}{\nobreakspace}
  \sphinxDUC{2500}{\sphinxunichar{2500}}
  \sphinxDUC{2502}{\sphinxunichar{2502}}
  \sphinxDUC{2514}{\sphinxunichar{2514}}
  \sphinxDUC{251C}{\sphinxunichar{251C}}
  \sphinxDUC{2572}{\textbackslash}
\fi
\usepackage{cmap}
\usepackage[T1]{fontenc}
\usepackage{amsmath,amssymb,amstext}
\usepackage{babel}



\usepackage{tgtermes}
\usepackage{tgheros}
\renewcommand{\ttdefault}{txtt}



\usepackage[Bjarne]{fncychap}
\usepackage{sphinx}

\fvset{fontsize=auto}
\usepackage{geometry}


% Include hyperref last.
\usepackage{hyperref}
% Fix anchor placement for figures with captions.
\usepackage{hypcap}% it must be loaded after hyperref.
% Set up styles of URL: it should be placed after hyperref.
\urlstyle{same}

\addto\captionsenglish{\renewcommand{\contentsname}{Contents:}}

\usepackage{sphinxmessages}
\setcounter{tocdepth}{1}



\title{GLVQ with Renyi Divergence}
\date{Jul 27, 2021}
\release{1.0}
\author{Foqrul Islam}
\newcommand{\sphinxlogo}{\vbox{}}
\renewcommand{\releasename}{Release}
\makeindex
\begin{document}

\pagestyle{empty}
\sphinxmaketitle
\pagestyle{plain}
\sphinxtableofcontents
\pagestyle{normal}
\phantomsection\label{\detokenize{index::doc}}



\chapter{Renyi}
\label{\detokenize{modules:renyi}}\label{\detokenize{modules::doc}}

\section{Renyi\_Classes and Functions}
\label{\detokenize{Renyi_final:renyi-classes-and-functions}}\label{\detokenize{Renyi_final::doc}}\phantomsection\label{\detokenize{Renyi_final:module-Renyi_final}}\index{module@\spxentry{module}!Renyi\_final@\spxentry{Renyi\_final}}\index{Renyi\_final@\spxentry{Renyi\_final}!module@\spxentry{module}}\index{GLVQ (class in Renyi\_final)@\spxentry{GLVQ}\spxextra{class in Renyi\_final}}

\begin{fulllineitems}
\phantomsection\label{\detokenize{Renyi_final:Renyi_final.GLVQ}}\pysiglinewithargsret{\sphinxbfcode{\sphinxupquote{class }}\sphinxcode{\sphinxupquote{Renyi\_final.}}\sphinxbfcode{\sphinxupquote{GLVQ}}}{\emph{\DUrole{n}{class\_prototype}}}{}
\sphinxAtStartPar
Bases: \sphinxcode{\sphinxupquote{object}}

\sphinxAtStartPar
Generlaized Learning Vector Quantization (GLVQ)
\begin{description}
\item[{Attributes:}] \leavevmode\begin{description}
\item[{class\_prototype:}] \leavevmode
\sphinxAtStartPar
The number of prototype of each class to be learned

\end{description}

\end{description}
\index{Renyi\_Divergence() (Renyi\_final.GLVQ method)@\spxentry{Renyi\_Divergence()}\spxextra{Renyi\_final.GLVQ method}}

\begin{fulllineitems}
\phantomsection\label{\detokenize{Renyi_final:Renyi_final.GLVQ.Renyi_Divergence}}\pysiglinewithargsret{\sphinxbfcode{\sphinxupquote{Renyi\_Divergence}}}{\emph{\DUrole{n}{data\_in}}, \emph{\DUrole{n}{prototypes}}}{}
\sphinxAtStartPar
Calculate the Renyi divergence between datapoints and prototypes.
\begin{quote}\begin{description}
\item[{Parameters}] \leavevmode\begin{itemize}
\item {} 
\sphinxAtStartPar
\sphinxstyleliteralstrong{\sphinxupquote{data\_in}} \textendash{} A n x m matrix of datapoints.

\item {} 
\sphinxAtStartPar
\sphinxstyleliteralstrong{\sphinxupquote{prototypes}} \textendash{} A n x m matrix of prototyes of each class

\end{itemize}

\item[{Returns}] \leavevmode
\sphinxAtStartPar
A n x m matrix with Renyi diverence between datapoints and prototypes

\end{description}\end{quote}

\end{fulllineitems}

\index{calculate\_d\_minus() (Renyi\_final.GLVQ method)@\spxentry{calculate\_d\_minus()}\spxextra{Renyi\_final.GLVQ method}}

\begin{fulllineitems}
\phantomsection\label{\detokenize{Renyi_final:Renyi_final.GLVQ.calculate_d_minus}}\pysiglinewithargsret{\sphinxbfcode{\sphinxupquote{calculate\_d\_minus}}}{\emph{\DUrole{n}{data\_labels}}, \emph{\DUrole{n}{prot\_labels}}, \emph{\DUrole{n}{pro\_types}}, \emph{\DUrole{n}{Renyi\_div}}}{}
\sphinxAtStartPar
Calculate the distance between data points and prototypes
\begin{quote}\begin{description}
\item[{Parameters}] \leavevmode\begin{itemize}
\item {} 
\sphinxAtStartPar
\sphinxstyleliteralstrong{\sphinxupquote{data\_labels}} \textendash{} A n\sphinxhyphen{}dimensional vector containing the labels for each datapoint.

\item {} 
\sphinxAtStartPar
\sphinxstyleliteralstrong{\sphinxupquote{prot\_labels}} \textendash{} A n\sphinxhyphen{}dimensional vector containing the labels for each prototype.

\item {} 
\sphinxAtStartPar
\sphinxstyleliteralstrong{\sphinxupquote{pro\_types}} \textendash{} A n x m matrix of prototyes of each class.

\item {} 
\sphinxAtStartPar
\sphinxstyleliteralstrong{\sphinxupquote{Renyi\_div}} \textendash{} A n x m matrix with Renyi divergence between datapoints and prototypes.

\end{itemize}

\item[{Returns}] \leavevmode
\sphinxAtStartPar
d\_minus: A n\sphinxhyphen{}dimensional vector having distance between datapoints and prototypes with different label.
w\_minus: A m x n matrix of nearest incorrect matching prototypes.
w\_minus\_index: A n\sphinxhyphen{}dimensional vector having the indices for nearest prototype to datapoints with different label.

\end{description}\end{quote}

\end{fulllineitems}

\index{calculate\_d\_plus() (Renyi\_final.GLVQ method)@\spxentry{calculate\_d\_plus()}\spxextra{Renyi\_final.GLVQ method}}

\begin{fulllineitems}
\phantomsection\label{\detokenize{Renyi_final:Renyi_final.GLVQ.calculate_d_plus}}\pysiglinewithargsret{\sphinxbfcode{\sphinxupquote{calculate\_d\_plus}}}{\emph{\DUrole{n}{data\_labels}}, \emph{\DUrole{n}{prot\_labels}}, \emph{\DUrole{n}{pro\_types}}, \emph{\DUrole{n}{Renyi\_div}}}{}
\sphinxAtStartPar
Calculate the distance between data points and prototypes
\begin{quote}\begin{description}
\item[{Parameters}] \leavevmode\begin{itemize}
\item {} 
\sphinxAtStartPar
\sphinxstyleliteralstrong{\sphinxupquote{data\_labels}} \textendash{} A n\sphinxhyphen{}dimensional vector containing the labels for each datapoint.

\item {} 
\sphinxAtStartPar
\sphinxstyleliteralstrong{\sphinxupquote{prot\_labels}} \textendash{} A n\sphinxhyphen{}dimensional vector containing the labels for each prototype.

\item {} 
\sphinxAtStartPar
\sphinxstyleliteralstrong{\sphinxupquote{pro\_types}} \textendash{} A n x m matrix of prototyes of each class.

\item {} 
\sphinxAtStartPar
\sphinxstyleliteralstrong{\sphinxupquote{Renyi\_div}} \textendash{} A n x m matrix with Renyi divergence between datapoints and prototypes.

\end{itemize}

\item[{Returns}] \leavevmode
\sphinxAtStartPar
d\_plus: A n\sphinxhyphen{}dimensional vector having distance between datapoints and prototypes with the same label.
w\_plus: A m x n matrix of nearest correct matching prototypes.
w\_plus\_index: A n\sphinxhyphen{}dimensional vector having the indices for nearest prototype to datapoints with the same label.

\end{description}\end{quote}

\end{fulllineitems}

\index{change\_in\_w\_minus() (Renyi\_final.GLVQ method)@\spxentry{change\_in\_w\_minus()}\spxextra{Renyi\_final.GLVQ method}}

\begin{fulllineitems}
\phantomsection\label{\detokenize{Renyi_final:Renyi_final.GLVQ.change_in_w_minus}}\pysiglinewithargsret{\sphinxbfcode{\sphinxupquote{change\_in\_w\_minus}}}{\emph{\DUrole{n}{data\_in}}, \emph{\DUrole{n}{learning\_rate}}, \emph{\DUrole{n}{classifier}}, \emph{\DUrole{n}{w\_minus}}, \emph{\DUrole{n}{w\_minus\_index}}, \emph{\DUrole{n}{d\_plus}}, \emph{\DUrole{n}{d\_minus}}}{}
\sphinxAtStartPar
Calculate the update of prototypes
\begin{quote}\begin{description}
\item[{Parameters}] \leavevmode\begin{itemize}
\item {} 
\sphinxAtStartPar
\sphinxstyleliteralstrong{\sphinxupquote{data\_in}} \textendash{} A n x m matrix of datapoints.

\item {} 
\sphinxAtStartPar
\sphinxstyleliteralstrong{\sphinxupquote{learning\_rate}} \textendash{} Learning rate (step size)

\item {} 
\sphinxAtStartPar
\sphinxstyleliteralstrong{\sphinxupquote{classifier}} \textendash{} Classify the vector as a winner or runner up prototype.

\item {} 
\sphinxAtStartPar
\sphinxstyleliteralstrong{\sphinxupquote{w\_minus}} \textendash{} A m x n matrix of nearest incorrect matching prototypes.

\item {} 
\sphinxAtStartPar
\sphinxstyleliteralstrong{\sphinxupquote{w\_minus\_index}} \textendash{} A n\sphinxhyphen{}dimensional vector having the indices for nearest prototype to datapoints with different label.

\item {} 
\sphinxAtStartPar
\sphinxstyleliteralstrong{\sphinxupquote{d\_plus}} \textendash{} A n\sphinxhyphen{}dimesional vector having the distance between datapoints and prototypes with the same label.

\item {} 
\sphinxAtStartPar
\sphinxstyleliteralstrong{\sphinxupquote{d\_minus}} \textendash{} A n\sphinxhyphen{}dimesional vector having the distance between datapoints and prototypes with different label.

\end{itemize}

\item[{Returns}] \leavevmode
\sphinxAtStartPar
The result of the updated prototype after calculating the update of prototypes with the same or different label.

\end{description}\end{quote}

\end{fulllineitems}

\index{change\_in\_w\_plus() (Renyi\_final.GLVQ method)@\spxentry{change\_in\_w\_plus()}\spxextra{Renyi\_final.GLVQ method}}

\begin{fulllineitems}
\phantomsection\label{\detokenize{Renyi_final:Renyi_final.GLVQ.change_in_w_plus}}\pysiglinewithargsret{\sphinxbfcode{\sphinxupquote{change\_in\_w\_plus}}}{\emph{\DUrole{n}{data\_in}}, \emph{\DUrole{n}{learning\_rate}}, \emph{\DUrole{n}{classifier}}, \emph{\DUrole{n}{w\_plus}}, \emph{\DUrole{n}{w\_plus\_index}}, \emph{\DUrole{n}{d\_plus}}, \emph{\DUrole{n}{d\_minus}}}{}
\sphinxAtStartPar
Calculate the update of prototypes
\begin{quote}\begin{description}
\item[{Parameters}] \leavevmode\begin{itemize}
\item {} 
\sphinxAtStartPar
\sphinxstyleliteralstrong{\sphinxupquote{data\_in}} \textendash{} A n x m matrix of datapoints.

\item {} 
\sphinxAtStartPar
\sphinxstyleliteralstrong{\sphinxupquote{learning\_rate}} \textendash{} Learning rate (step size)

\item {} 
\sphinxAtStartPar
\sphinxstyleliteralstrong{\sphinxupquote{classifier}} \textendash{} Classify the vector as a winner or runner up prototype.

\item {} 
\sphinxAtStartPar
\sphinxstyleliteralstrong{\sphinxupquote{w\_plus}} \textendash{} A m x n matrix of nearest correct matching prototypes.

\item {} 
\sphinxAtStartPar
\sphinxstyleliteralstrong{\sphinxupquote{w\_plus\_index}} \textendash{} A n\sphinxhyphen{}dimensional vector having the indices for nearest prototype to datapoints with the same label.

\item {} 
\sphinxAtStartPar
\sphinxstyleliteralstrong{\sphinxupquote{d\_plus}} \textendash{} A n\sphinxhyphen{}dimesional vector having the distance between datapoints and prototypes with the same label.

\item {} 
\sphinxAtStartPar
\sphinxstyleliteralstrong{\sphinxupquote{d\_minus}} \textendash{} A n\sphinxhyphen{}dimesional vector having the distance between datapoints and prototypes with different label.

\end{itemize}

\item[{Returns}] \leavevmode
\sphinxAtStartPar
The result of the updated prototype after calculating the update of prototypes with the same or different label.

\end{description}\end{quote}

\end{fulllineitems}

\index{create\_data\_prototype() (Renyi\_final.GLVQ method)@\spxentry{create\_data\_prototype()}\spxextra{Renyi\_final.GLVQ method}}

\begin{fulllineitems}
\phantomsection\label{\detokenize{Renyi_final:Renyi_final.GLVQ.create_data_prototype}}\pysiglinewithargsret{\sphinxbfcode{\sphinxupquote{create\_data\_prototype}}}{\emph{\DUrole{n}{data\_in}}, \emph{\DUrole{n}{data\_labels}}, \emph{\DUrole{n}{class\_prototype}}}{}
\sphinxAtStartPar
Calculate prototypes with labels. The calculation is based on the mean or at random based on the
prototype in each class.
\begin{quote}\begin{description}
\item[{Parameters}] \leavevmode\begin{itemize}
\item {} 
\sphinxAtStartPar
\sphinxstyleliteralstrong{\sphinxupquote{data\_in}} \textendash{} A n x m matrix of datapoints

\item {} 
\sphinxAtStartPar
\sphinxstyleliteralstrong{\sphinxupquote{data\_labels}} \textendash{} A n\sphinxhyphen{}dimensional vector containing the labels for each datapoint

\item {} 
\sphinxAtStartPar
\sphinxstyleliteralstrong{\sphinxupquote{class\_prototype}} \textendash{} The number of prototype in each class to be learned. If the number of prototypes

\end{itemize}

\end{description}\end{quote}

\sphinxAtStartPar
in each class in 1, the prototypes aer assigned in the mean position, otherwise it is assigned at random.
\begin{quote}\begin{description}
\item[{Returns}] \leavevmode
\sphinxAtStartPar
\begin{description}
\item[{pro\_labels:}] \leavevmode
\sphinxAtStartPar
A n\sphinxhyphen{}dimensional vector having the labels for every prototype

\item[{Prototypes:}] \leavevmode
\sphinxAtStartPar
A n x m matrix of prototypes for the training purpose.

\end{description}


\end{description}\end{quote}

\end{fulllineitems}

\index{data\_predictor() (Renyi\_final.GLVQ method)@\spxentry{data\_predictor()}\spxextra{Renyi\_final.GLVQ method}}

\begin{fulllineitems}
\phantomsection\label{\detokenize{Renyi_final:Renyi_final.GLVQ.data_predictor}}\pysiglinewithargsret{\sphinxbfcode{\sphinxupquote{data\_predictor}}}{\emph{\DUrole{n}{input\_value}}}{}
\sphinxAtStartPar
The prediction of the labels for the data. The data are represented by the test\sphinxhyphen{}to\sphinxhyphen{}training distance matrix. Every
datapoint will be assigned to the closest prototype.
\begin{quote}\begin{description}
\item[{Parameters}] \leavevmode
\sphinxAtStartPar
\sphinxstyleliteralstrong{\sphinxupquote{input\_value}} \textendash{} A n x m matrix of distances from the test to the training datapoints.

\item[{Returns}] \leavevmode
\sphinxAtStartPar
y\_label \sphinxhyphen{} A n\sphinxhyphen{}dimensional vector having the predicted labels for each and every datapoint.

\end{description}\end{quote}

\end{fulllineitems}

\index{do\_processing() (Renyi\_final.GLVQ method)@\spxentry{do\_processing()}\spxextra{Renyi\_final.GLVQ method}}

\begin{fulllineitems}
\phantomsection\label{\detokenize{Renyi_final:Renyi_final.GLVQ.do_processing}}\pysiglinewithargsret{\sphinxbfcode{\sphinxupquote{do\_processing}}}{\emph{\DUrole{n}{data\_in}}, \emph{\DUrole{n}{data\_labels}}, \emph{\DUrole{n}{learning\_rate}}, \emph{\DUrole{n}{epochs}}}{}
\sphinxAtStartPar
Main function to train the algorithm
\begin{quote}\begin{description}
\item[{Parameters}] \leavevmode\begin{itemize}
\item {} 
\sphinxAtStartPar
\sphinxstyleliteralstrong{\sphinxupquote{data\_in}} \textendash{} A m x n matrix of distance.

\item {} 
\sphinxAtStartPar
\sphinxstyleliteralstrong{\sphinxupquote{data\_labels}} \textendash{} A m\sphinxhyphen{}dimensional vector containing the labels for each datapoint.

\item {} 
\sphinxAtStartPar
\sphinxstyleliteralstrong{\sphinxupquote{learning\_rate}} \textendash{} Learning rate (step size).

\item {} 
\sphinxAtStartPar
\sphinxstyleliteralstrong{\sphinxupquote{epochs}} \textendash{} The maximum number of optimization iterations.

\end{itemize}

\item[{Returns}] \leavevmode
\sphinxAtStartPar
A n\sphinxhyphen{}dimensional updated prototype vector.

\end{description}\end{quote}

\end{fulllineitems}

\index{new\_proto\_types (Renyi\_final.GLVQ attribute)@\spxentry{new\_proto\_types}\spxextra{Renyi\_final.GLVQ attribute}}

\begin{fulllineitems}
\phantomsection\label{\detokenize{Renyi_final:Renyi_final.GLVQ.new_proto_types}}\pysigline{\sphinxbfcode{\sphinxupquote{new\_proto\_types}}\sphinxbfcode{\sphinxupquote{ = array({[}{]}, dtype=float64)}}}
\end{fulllineitems}

\index{normalize\_data() (Renyi\_final.GLVQ method)@\spxentry{normalize\_data()}\spxextra{Renyi\_final.GLVQ method}}

\begin{fulllineitems}
\phantomsection\label{\detokenize{Renyi_final:Renyi_final.GLVQ.normalize_data}}\pysiglinewithargsret{\sphinxbfcode{\sphinxupquote{normalize\_data}}}{\emph{\DUrole{n}{data\_in}}}{}
\sphinxAtStartPar
Normalizing the data so that the data is between the range 0 and 1 and also we can further the data as
a probability distribution.
\begin{quote}\begin{description}
\item[{Parameters}] \leavevmode
\sphinxAtStartPar
\sphinxstyleliteralstrong{\sphinxupquote{data\_in}} \textendash{} A nx m matrix of our input data.

\item[{Returns}] \leavevmode
\sphinxAtStartPar
Data\_normalized: A n x m matrix with values between 0 and 1

\end{description}\end{quote}

\end{fulllineitems}

\index{plot() (Renyi\_final.GLVQ method)@\spxentry{plot()}\spxextra{Renyi\_final.GLVQ method}}

\begin{fulllineitems}
\phantomsection\label{\detokenize{Renyi_final:Renyi_final.GLVQ.plot}}\pysiglinewithargsret{\sphinxbfcode{\sphinxupquote{plot}}}{\emph{\DUrole{n}{data\_in}}, \emph{\DUrole{n}{data\_labels}}, \emph{\DUrole{n}{prot\_types}}, \emph{\DUrole{n}{prot\_labels}}}{}
\sphinxAtStartPar
Visualizing the data and prototypes in a scatter plot.
\begin{quote}\begin{description}
\item[{Parameters}] \leavevmode\begin{itemize}
\item {} 
\sphinxAtStartPar
\sphinxstyleliteralstrong{\sphinxupquote{data\_in}} \textendash{} A n x m matrix of datapoints.

\item {} 
\sphinxAtStartPar
\sphinxstyleliteralstrong{\sphinxupquote{data\_labels}} \textendash{} A n\sphinxhyphen{}dimensional vector containing the labels for each datapoint.

\item {} 
\sphinxAtStartPar
\sphinxstyleliteralstrong{\sphinxupquote{prot\_types}} \textendash{} A n x m matrix of prototyes of each class.

\item {} 
\sphinxAtStartPar
\sphinxstyleliteralstrong{\sphinxupquote{prot\_labels}} \textendash{} A n\sphinxhyphen{}dimensional vector containing the labels for each prototype.

\end{itemize}

\end{description}\end{quote}

\end{fulllineitems}

\index{prototype\_data\_labels (Renyi\_final.GLVQ attribute)@\spxentry{prototype\_data\_labels}\spxextra{Renyi\_final.GLVQ attribute}}

\begin{fulllineitems}
\phantomsection\label{\detokenize{Renyi_final:Renyi_final.GLVQ.prototype_data_labels}}\pysigline{\sphinxbfcode{\sphinxupquote{prototype\_data\_labels}}\sphinxbfcode{\sphinxupquote{ = array({[}{]}, dtype=float64)}}}
\end{fulllineitems}

\index{sigmoid\_calc() (Renyi\_final.GLVQ method)@\spxentry{sigmoid\_calc()}\spxextra{Renyi\_final.GLVQ method}}

\begin{fulllineitems}
\phantomsection\label{\detokenize{Renyi_final:Renyi_final.GLVQ.sigmoid_calc}}\pysiglinewithargsret{\sphinxbfcode{\sphinxupquote{sigmoid\_calc}}}{\emph{\DUrole{n}{j}}, \emph{\DUrole{n}{beta}\DUrole{o}{=}\DUrole{default_value}{10}}}{}
\sphinxAtStartPar
Calculate the sigmoid activation function.
\begin{quote}\begin{description}
\item[{Parameters}] \leavevmode\begin{itemize}
\item {} 
\sphinxAtStartPar
\sphinxstyleliteralstrong{\sphinxupquote{j}} \textendash{} A n x m matrix of datapoints or any value

\item {} 
\sphinxAtStartPar
\sphinxstyleliteralstrong{\sphinxupquote{beta}} \textendash{} The value of the parameter is taken as 10

\end{itemize}

\item[{Returns}] \leavevmode
\sphinxAtStartPar
It returns values in the range 0 to 1

\end{description}\end{quote}

\end{fulllineitems}


\end{fulllineitems}



\chapter{Installation guide}
\label{\detokenize{index:installation-guide}}
\sphinxAtStartPar
The following requirements need to be fulfilled to run the program:
\begin{itemize}
\item {} 
\sphinxAtStartPar
python with minimum version 2.8

\item {} 
\sphinxAtStartPar
numpy with minimum version 1.19.0

\item {} 
\sphinxAtStartPar
matplotlib

\item {} 
\sphinxAtStartPar
Scikit Learn.

\end{itemize}


\section{Execution}
\label{\detokenize{index:execution}}\begin{itemize}
\item {} 
\sphinxAtStartPar
Copy the file in a folder

\item {} 
\sphinxAtStartPar
Install the required packages mentioned in the ‘Installation guide’

\item {} 
\sphinxAtStartPar
Run the program

\end{itemize}


\chapter{Indices and tables}
\label{\detokenize{index:indices-and-tables}}\begin{itemize}
\item {} 
\sphinxAtStartPar
\DUrole{xref,std,std-ref}{genindex}

\item {} 
\sphinxAtStartPar
\DUrole{xref,std,std-ref}{modindex}

\item {} 
\sphinxAtStartPar
\DUrole{xref,std,std-ref}{search}

\end{itemize}


\renewcommand{\indexname}{Python Module Index}
\begin{sphinxtheindex}
\let\bigletter\sphinxstyleindexlettergroup
\bigletter{r}
\item\relax\sphinxstyleindexentry{Renyi\_final}\sphinxstyleindexpageref{Renyi_final:\detokenize{module-Renyi_final}}
\end{sphinxtheindex}

\renewcommand{\indexname}{Index}
\printindex
\end{document}